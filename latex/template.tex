
 \documentclass{article}[10pt]
\usepackage{changepage} 
\usepackage[T1]{fontenc}
\usepackage{sectsty}
\usepackage{fancyhdr}
\usepackage[sfdefault]{roboto}
\usepackage{hyperref}
\usepackage[utf8]{inputenc}
\usepackage{lipsum}

\hypersetup{
    colorlinks,
    citecolor=black,
    filecolor=black,
    linkcolor=black,
    urlcolor=black
}
\pagenumbering{arabic}
\sectionfont{\fontsize{30}{15}\fontseries{t}\selectfont}
\subsectionfont{\fontsize{14}{15}\fontseries{l}\selectfont}

\renewcommand*\contentsname{Szkolenia:}

\pagestyle{fancy}
\fancyhead{}

\fancyfoot{}
\fancyfoot[R]{\thepage}

\renewcommand{\headrulewidth}{0pt}
\rhead{\fontsize{14}{12} \selectfont header (sam of every page)}

\setcounter{secnumdepth}{0} % sections are level 1

\begin{document}

	\pagestyle{empty} %get rid of header/footer for toc page
    \tableofcontents %put toc in
    \cleardoublepage %start new page
    \pagestyle{fancy} % put headers/footers back on
    \setcounter{page}{1} %reset the page counter

	\newpage
    
	\section{JavaScript – kurs podstawowy (dla programistów)}

	\subsection*{description:}
	\begin{adjustwidth}{2cm}{}
		Celem szkolenia jest przekazanie wiedzy na temat mechanizmów działania języka JavaScript.
Uczestnicy nauczą się budować aplikacje w oparciu o najnowsze standardy oraz poznają funkcyjne i obiektowe możliwości języka. Dostaną również szereg informacji na temat tego w jaki sposób pisać skalowalny kod i jak używać współczesnych wzorców projektowych takich jak
MVC, MVVM, Flux. Dzięki przekazanej wiedzy uczestnicy będą rozumieć działanie języka co pozwoli na szybki dalszy rozwój oraz unikanie najczęstszych błędów przy tworzeniu aplikacji internetowych.
	\end{adjustwidth}
	\subsection*{profile:}
\begin{adjustwidth}{2cm}{}
	
Szkolenie przeznaczone jest dla osób, znających dowolny język programowania. Dla tych, którzy dopiero zaczynają swoją przygodę z programowaniem sporządzony został kurs - Wstęp do programowania z JavaScript

\end{adjustwidth}
	\subsection*{duration:}
\begin{adjustwidth}{2cm}{}
	3
\end{adjustwidth}

	\subsection*{list:}
\begin{adjustwidth}{1.5cm}{}
	\begin{itemize}




















	\end{itemize}
\end{adjustwidth}

	\subsection*{skills:}
\begin{adjustwidth}{2cm}{}
	
Po zakończonym szkoleniu uczestnicy będą znać podstawy tworzenia aplikacji internetowych. Będą znali różne podejścia programistyczne i będą wiedzieć w jaki sposób będą mogli w prosty sposób rozwijać się dalej w obranym kierunku.

\end{adjustwidth}

\newpage


    
	\section{JavaScript Zaawansowany}

	\subsection*{description:}
	\begin{adjustwidth}{2cm}{}
		
Celem szkolenia jest przekazanie informacji na temat zaawansowanych mechanizmów działania języka JavaScript. Uczestnicy dowiedzą się w jaki sposób mogą tworzyć Web Components bez wsparcia żadnego frameworka, poznają sposób działania nowoczesnych featurów języka takich jak np. generatory, iterables, proxies, pogłębią informacje na temat mechanizmów takich jak prototypy, zmiana kontekstu jak również dostaną szereg wiadomości na temat asynchroniczności, funkcyjnego podejścia i silników przeglądarek. Nabyta wiedza pozwoli pisać bardziej ekspresywny, czysty i wydajny kod.
	\end{adjustwidth}
	\subsection*{profile:}
\begin{adjustwidth}{2cm}{}
	
Szkolenie adresowane jest do programistów, którzy znają podstawy języka JavaScript  - posiadają wiedzę na poziomie ukończonego szkolenia JavaScript kurs podstawowy (dla programistów).
\end{adjustwidth}
	\subsection*{duration:}
\begin{adjustwidth}{2cm}{}
	3
\end{adjustwidth}

	\subsection*{list:}
\begin{adjustwidth}{1.5cm}{}
	\begin{itemize}



























	\end{itemize}
\end{adjustwidth}

	\subsection*{skills:}
\begin{adjustwidth}{2cm}{}
	
Po ukończonym szkoleniu uczestnicy będą znać szczegółowo zasady działania prototypów oraz innych nisko poziomowych mechanizmów w JavaScripcie, umieć wykorzystać najnowocześniejsze featury języka oraz rozumieć nowoczesne wzorce architektoniczne.

\end{adjustwidth}

\newpage


    
	\section{TypeScript}

	\subsection*{description:}
	\begin{adjustwidth}{2cm}{}
		
Celem szkolenia jest przekazanie wiedzy na temat języka TypeScript. Uczestnicy dowiedzą się w jaki sposób mogą polepszyć jakość kodu dodając do niego typy, również w postaci skomplikowanych struktur. Nauczą się również korzystać featurów, które udostępnia TypeScript - nauczą się np. w jaki sposób pisać dekoratory klas, metod oraz pól. Dzięki temu szkoleniu Dzięki temu szkoleniu uczestnicy będą wiedzieć w jaki sposób zaadoptować techniki znane z klasycznych języków programowania takich jak Java czy C# do pisania aplikacji frontendowych.

	\end{adjustwidth}
	\subsection*{profile:}
\begin{adjustwidth}{2cm}{}
	
Szkolenie adresowane jest do programistów, którzy znają podstawy języka JavaScript  - posiadają wiedzę na poziomie ukończonego szkolenia JavaScript kurs podstawowy (dla programistów).
\end{adjustwidth}
	\subsection*{duration:}
\begin{adjustwidth}{2cm}{}
	8
\end{adjustwidth}

	\subsection*{list:}
\begin{adjustwidth}{1.5cm}{}
	\begin{itemize}

























	\end{itemize}
\end{adjustwidth}

	\subsection*{skills:}
\begin{adjustwidth}{2cm}{}
	
Po zakończonym szkoleniu uczestnicy będą swobodnie posługiwać się językiem TypeScript, będą wiedzieli w jaki sposób można pisać aplikacje frontendowe używając praktyk znanych z klasycznych języków obiektowych takich jak Java czy C#.



\end{adjustwidth}

\newpage


    
	\section{Angular - kurs podstawowy}

	\subsection*{description:}
	\begin{adjustwidth}{2cm}{}
		
Celem szkolenia jest nauczenie uczestników w jaki sposób budować aplikacje w oparciu o framework Angular. Uczestnicy poznają mechanizmy frameworka co pozwoli im na samodzielną pracę w tym środowisku. Dodatkowo poznają struktury architektoniczne, które mogą zostać zaimplementowane w angularowych aplikacjach. 

	\end{adjustwidth}
	\subsection*{profile:}
\begin{adjustwidth}{2cm}{}
	
Szkolenie adresowane jest do programistów znających język JavaScript.
\end{adjustwidth}
	\subsection*{duration:}
\begin{adjustwidth}{2cm}{}
	3
\end{adjustwidth}

	\subsection*{list:}
\begin{adjustwidth}{1.5cm}{}
	\begin{itemize}

























	\end{itemize}
\end{adjustwidth}

	\subsection*{skills:}
\begin{adjustwidth}{2cm}{}
	
Po ukończonym szkoleniu uczestnicy będą potrafili tworzyć aplikacje używając frameworka Angular. Będą posiadali informacje na temat możliwych architektur aplikacji oraz głównych mechanizmów działania frameworka. 

\end{adjustwidth}

\newpage


    
	\section{Angular zaawansowany}

	\subsection*{description:}
	\begin{adjustwidth}{2cm}{}
		
Celem szkolenia jest przekazanie informacji na temat zaawansowanych mechanizmów działania frameworka Angular. Uczestnicy dowiedzą się w jaki sposób tworzyć m.in. własne dyrektywy, dynamiczne komponenty, komponenty reaktywne, zagnieżdżony routing, dostaną szereg informacji na temat performance’u i możliwych memory leak’ów oraz dowiedzą się w jaki sposób mogą użyć Angulara razem z nowoczesnymi podejściami do zarządzania stanem takimi jak Flux czy Model-View-Intent.
Uczestnicy poznają także w szczegółach zasadę pracy z Observables i nauczą się operować na strumieniach danych.

	\end{adjustwidth}
	\subsection*{profile:}
\begin{adjustwidth}{2cm}{}
	
Szkolenie adresowane jest do programistów znających podstawy frameworka Angular.
\end{adjustwidth}
	\subsection*{duration:}
\begin{adjustwidth}{2cm}{}
	3
\end{adjustwidth}

	\subsection*{list:}
\begin{adjustwidth}{1.5cm}{}
	\begin{itemize}




















	\end{itemize}
\end{adjustwidth}

	\subsection*{skills:}
\begin{adjustwidth}{2cm}{}
	
Po ukończonym kursie uczestnicy będą znali zaawansowane mechanizmy frameworka. Będą wiedzieli jak optymalizować swoje aplikacje, jak pisać czysty, wydajny kod i jak tworzyć przejrzyste architektury angularowych aplikacji.



\end{adjustwidth}

\newpage


    
	\section{Angular - kompedium wiedzy (TS, RxJS)}

	\subsection*{description:}
	\begin{adjustwidth}{2cm}{}
		
Celem szkolenia jest nauczenie uczestników w jaki sposób budować aplikacje w oparciu o framework Angular. Uczestnicy poznają mechanizmy frameworka co pozwoli im na samodzielną pracę w tym środowisku. Dodatkowo poznają struktury architektoniczne, które mogą zostać zaimplementowane w angularowych aplikacjach. 

	\end{adjustwidth}
	\subsection*{profile:}
\begin{adjustwidth}{2cm}{}
	
Szkolenie adresowane jest do programistów znających język JavaScript, chcących się nauczyć Reacta od podstaw.
\end{adjustwidth}
	\subsection*{duration:}
\begin{adjustwidth}{2cm}{}
	5
\end{adjustwidth}

	\subsection*{list:}
\begin{adjustwidth}{1.5cm}{}
	\begin{itemize}













































	\end{itemize}
\end{adjustwidth}

	\subsection*{skills:}
\begin{adjustwidth}{2cm}{}
	
Po ukończonym kursie uczestnicy będą znali zaawansowane mechanizmy frameworka. Będą wiedzieli jak optymalizować swoje aplikacje, jak pisać czysty, wydajny kod i jak tworzyć przejrzyste architektury angularowych aplikacji.



\end{adjustwidth}

\newpage


    
	\section{React.js}

	\subsection*{description:}
	\begin{adjustwidth}{2cm}{}
		
Celem szkolenia jest nauczenie od podstaw biblioteki React.js oraz praktycznego jej wykorzystania do tworzenia nowoczesnych aplikacji internetowych. Uczestnicy dowiedzą się w jaki sposób dzielić aplikacje na komponenty oraz na czym polegają współczesne architektury do zarządzania stanem aplikacji - takie jak React hooks, Redux czy MobX i czym różnią się od klasycznych podejść  typu Model-View-Controller.
Po ukończonym szkoleniu uczestnicy będą w stanie samodzielnie pisać aplikacje wykorzystujące bibliotekę React. 
	\end{adjustwidth}
	\subsection*{profile:}
\begin{adjustwidth}{2cm}{}
	
Szkolenie adresowane jest do programistów znających język JavaScript, chcących się nauczyć Reacta od podstaw.
\end{adjustwidth}
	\subsection*{duration:}
\begin{adjustwidth}{2cm}{}
	2
\end{adjustwidth}

	\subsection*{list:}
\begin{adjustwidth}{1.5cm}{}
	\begin{itemize}





















	\end{itemize}
\end{adjustwidth}

	\subsection*{skills:}
\begin{adjustwidth}{2cm}{}
	
Po ukończonym szkoleniu uczestnicy będą znać zasady działania biblioteki React i będą wiedzieć jak użyć tej biblioteki razem z architekturami Fluxowymi takimi jak Redux oraz MobX. oraz z React hooks Zdobędą informacje na temat dobrych praktyk, podejścia funkcyjnego oraz tworzenia przejrzystych struktur aplikacji reactowych.

\end{adjustwidth}

\newpage


    
	\section{WebWorkers i Progressive Web Apps}

	\subsection*{description:}
	\begin{adjustwidth}{2cm}{}
		Celem szkolenia jest przekazać wiedzę na temat technologii związanych z progressive web applications. Uczestnicy dowiedzą się w jaki sposób budować skalowalne aplikacje, które z poziomu użytkownika będą wyglądały jak aplikacje natywne
	\end{adjustwidth}
	\subsection*{profile:}
\begin{adjustwidth}{2cm}{}
	Szkolenie adresowane jest dla osób, które znają JS w stopniu średniozaawansowanym
\end{adjustwidth}
	\subsection*{duration:}
\begin{adjustwidth}{2cm}{}
	2
\end{adjustwidth}

	\subsection*{list:}
\begin{adjustwidth}{1.5cm}{}
	\begin{itemize}







	\end{itemize}
\end{adjustwidth}

	\subsection*{skills:}
\begin{adjustwidth}{2cm}{}
	undefined
\end{adjustwidth}

\newpage


    
	\section{Node.js kurs podstawowy (dla programistów front-end)}

	\subsection*{description:}
	\begin{adjustwidth}{2cm}{}
		
Celem szkolenia jest przekazanie wiedzy osobom, które do tej pory pracowały tylko we frontendzie, na temat tego w jaki sposób język programowania z którym mają do czynienia na co dzień (JavaScript) może zostać użyty na serwerze. Uczestnicy poznają środowisko, zrozumieją jego mechanikę oraz nauczą się pisać proste aplikacje po stronie serwera. Dzięki praktycznym ćwiczeniom nauczą się pracy z bazami danych oraz zarządzania kolejkami asynchronicznych funkcji.


	\end{adjustwidth}
	\subsection*{profile:}
\begin{adjustwidth}{2cm}{}
	
Szkolenie adresowane jest do programistów aplikacji frontendowych nie mających doświadczenia w tworzeniu aplikacji backendowych.
\end{adjustwidth}
	\subsection*{duration:}
\begin{adjustwidth}{2cm}{}
	2
\end{adjustwidth}

	\subsection*{list:}
\begin{adjustwidth}{1.5cm}{}
	\begin{itemize}













	\end{itemize}
\end{adjustwidth}

	\subsection*{skills:}
\begin{adjustwidth}{2cm}{}
	
Po skończonym szkoleniu uczestnicy będa znali założenia środowiska Node.js, jego architekturę oraz nauczą się tworzenia prostych programów komunikujących się z bazami danych.

\end{adjustwidth}

\newpage


    
	\section{Wstęp do programowania reaktywnego}

	\subsection*{description:}
	\begin{adjustwidth}{2cm}{}
		
Celem szkolenia jest przekazanie wiedzy na temat reaktywnego programowania na podstawie budowania aplikacji w oparciu o bibliotekę RxJS. Uczestnicy poznają różnicę między podejściem imperatywnym i reaktywnym i zrozumieją wady i zalety każdego z nich. Po ukończeniu szkolenia będą w stanie samodzielnie zaimplementować reaktywne struktury i będą wiedzieć w których miejscach aplikacji takie podejście może bardzo poprawić działanie aplikacji.

	\end{adjustwidth}
	\subsection*{profile:}
\begin{adjustwidth}{2cm}{}
	
Szkolenie adresowane jest do programistów znających dowolny język programowania.
\end{adjustwidth}
	\subsection*{duration:}
\begin{adjustwidth}{2cm}{}
	8
\end{adjustwidth}

	\subsection*{list:}
\begin{adjustwidth}{1.5cm}{}
	\begin{itemize}










	\end{itemize}
\end{adjustwidth}

	\subsection*{skills:}
\begin{adjustwidth}{2cm}{}
	
Po ukończony szkoleniu uczestnicy będą w stanie tworzyć reaktywne oprogramowanie i będą rozumieli w jakich przypadkach, tego typu podejście może zdecydowanie polepszyć jakość programu.
\end{adjustwidth}

\newpage


    
	\section{Object-oriented programming
w JavaScript}

	\subsection*{description:}
	\begin{adjustwidth}{2cm}{}
		
Celem szkolenia jest przekazanie informacji na temat mechanizmów języka JavaScript pozwalających na tworzenie skalowalnego i wydajnego oprogramowania stosując techniki OOP.
	\end{adjustwidth}
	\subsection*{profile:}
\begin{adjustwidth}{2cm}{}
	
Szkolenie adresowane jest do programistów, którzy znają podstawy języka JavaScript  - posiadają wiedzę na poziomie ukończonego szkolenia JavaScript kurs podstawowy (dla programistów).
\end{adjustwidth}
	\subsection*{duration:}
\begin{adjustwidth}{2cm}{}
	2
\end{adjustwidth}

	\subsection*{list:}
\begin{adjustwidth}{1.5cm}{}
	\begin{itemize}


























	\end{itemize}
\end{adjustwidth}

	\subsection*{skills:}
\begin{adjustwidth}{2cm}{}
	
Po ukończonym szkoleniu uczestnicy będą znać szczegółowo zasady działania mechanizmów pomagających w tworzeniu programów w JavaScript w podejściu obiektowym
\end{adjustwidth}

\newpage


    
	\section{Programowanie funkcyjne w JavaScript}

	\subsection*{description:}
	\begin{adjustwidth}{2cm}{}
		 Paradygmat programowania funkcyjnego jest coraz cześciej, używany w JavaScripcie. Czyste funkcje, niemutowalność stanu, składanie funkcji, funkcje wyższego rzędu, monady... to wszystko zagadnienia, które z łatwością mogą zostać zaimplementowane w JavaScripcie.
 Zrozumienie podejścia funkcyjnego daje programiście dużo większe możliwości, a kod zaimplementowany w ten sposób jest zdecydowanie bardziej ekspresywny, skalowalny i łatwiejszy do testowania.

	\end{adjustwidth}
	\subsection*{profile:}
\begin{adjustwidth}{2cm}{}
	
Szkolenie adresowane jest do programistów, którzy znają podstawy języka JavaScript  - posiadają wiedzę na poziomie ukończonego szkolenia JavaScript kurs podstawowy (dla programistów).
\end{adjustwidth}
	\subsection*{duration:}
\begin{adjustwidth}{2cm}{}
	8
\end{adjustwidth}

	\subsection*{list:}
\begin{adjustwidth}{1.5cm}{}
	\begin{itemize}











	\end{itemize}
\end{adjustwidth}

	\subsection*{skills:}
\begin{adjustwidth}{2cm}{}
	
Po zakończonym szkoleniu uczestnicy będą znać funkcyjne możliwości języka JavaScript, umieć je wykorzystać w praktyce oraz będą świadomi jakie wady i zalety takie podejście za sobą niesie.
\end{adjustwidth}

\newpage


    
	\section{Architektura aplikacji frontendowych}

	\subsection*{description:}
	\begin{adjustwidth}{2cm}{}
		Celem szkolenia jest nauczenie uczestników w jaki sposób tworzyć skalowalne architektury aplikacji oraz jak wyglądają nowoczesne narzędzia JS umożliwiające tworzenie i utrzymywanie dużych systemów frontendowych.
	\end{adjustwidth}
	\subsection*{profile:}
\begin{adjustwidth}{2cm}{}
	
Szkolenie adresowane jest do programistów znających język JavaScript.
\end{adjustwidth}
	\subsection*{duration:}
\begin{adjustwidth}{2cm}{}
	3
\end{adjustwidth}

	\subsection*{list:}
\begin{adjustwidth}{1.5cm}{}
	\begin{itemize}































	\end{itemize}
\end{adjustwidth}

	\subsection*{skills:}
\begin{adjustwidth}{2cm}{}
	
Po ukończonym szkoleniu uczestnicy będą znali zagadnienia związane z architekturą aplikacji. Poznają również szereg narzędzi, które pozwalają na dużo łatwiejsze pisanie programów frontendowych 


\end{adjustwidth}

\newpage


    
	\section{ES6+}

	\subsection*{description:}
	\begin{adjustwidth}{2cm}{}
		
Celem szkolenia jest przekazanie wiedzy na temat nowych featurów języka JavaScript, które zostały zaimplementowane w standardach EcmaScript nowszych niż EcmaScript 5. Uczestnicy nauczą się podstawowych mechanizmów takich jak zmienne blokowe jak również tych bardziej skomplikowanych takich jak iteratory, generatory czy metaprogrmowanie przy użyciu proxy

	\end{adjustwidth}
	\subsection*{profile:}
\begin{adjustwidth}{2cm}{}
	
Szkolenie adresowane jest do programistów znających język JavaScript w wersji EcmaScript 5 chcących poznać możliwości języka, które zostały dodane w standardach EcmaScript 2015 i kolejnych (czyli ES6+ ).
\end{adjustwidth}
	\subsection*{duration:}
\begin{adjustwidth}{2cm}{}
	8
\end{adjustwidth}

	\subsection*{list:}
\begin{adjustwidth}{1.5cm}{}
	\begin{itemize}
























	\end{itemize}
\end{adjustwidth}

	\subsection*{skills:}
\begin{adjustwidth}{2cm}{}
	
Po zakończonym szkoleniu uczestnicy będą sprawnie posługiwać się nowymi featurami języka JavaScript.
\end{adjustwidth}

\newpage


    
	\section{Testy automatyczne aplikacji webowych dla programistów
}

	\subsection*{description:}
	\begin{adjustwidth}{2cm}{}
		
Celem szkolenia jest nauczenie uczestników czym jest testowalny kod, jak pisać testy jednostkowe oraz testy end-to-end. Uczestnicy dowiedzą się również czym jest podejście typu Test-Driven Development (TDD),  Behaviour Driven Development (BDD) oraz jakie są wady i zalety podejść tego typu.

	\end{adjustwidth}
	\subsection*{profile:}
\begin{adjustwidth}{2cm}{}
	
Podstawowa wiedza z zakresu programowania.
\end{adjustwidth}
	\subsection*{duration:}
\begin{adjustwidth}{2cm}{}
	2
\end{adjustwidth}

	\subsection*{list:}
\begin{adjustwidth}{1.5cm}{}
	\begin{itemize}
































	\end{itemize}
\end{adjustwidth}

	\subsection*{skills:}
\begin{adjustwidth}{2cm}{}
	
Po ukończonym szkoleniu uczestnicy będą posiadać wiedzę na temat pisania testów jednostkowych oraz testów typu end-to-end. Będą wiedzieć jak pisać testowalny kod co zwiększy jego przejrzystość i jakość. 



\end{adjustwidth}

\newpage


    
	\section{Wstęp do programowania z JavaScript}

	\subsection*{description:}
	\begin{adjustwidth}{2cm}{}
		Celem szkolenia jest nauczenie uczestników podstaw programowania w języku JavaScript. Jest to kurs dla osób, które dopiero chcą zacząć swoją przygodę z programowaniem. Uczestnicy poznają podstawy programowania takie jak instrukcje warunkowe, zmienne czy pętle, zrozumieją na czym polegają różne podejścia programistyczne i dostaną szereg narzędzi pozwalających na późniejszy samodzielny rozwój.
	\end{adjustwidth}
	\subsection*{profile:}
\begin{adjustwidth}{2cm}{}
	Szkolenie adresowane jest dla osób, które nie potrafią jeszcze programować w żadnym języku.
\end{adjustwidth}
	\subsection*{duration:}
\begin{adjustwidth}{2cm}{}
	3
\end{adjustwidth}

	\subsection*{list:}
\begin{adjustwidth}{1.5cm}{}
	\begin{itemize}










	\end{itemize}
\end{adjustwidth}

	\subsection*{skills:}
\begin{adjustwidth}{2cm}{}
	undefined
\end{adjustwidth}

\newpage


    
	\section{Wykorzystywanie kodu Java, Rust, Cpp w przeglądarce - wprowadzenie do WebAssembly}

	\subsection*{description:}
	\begin{adjustwidth}{2cm}{}
		W czasie szkolenia uczesnticy zdobędą szereg informacji na temat języka webassembly. Webassembly pozwala na użycie kodu źródłowego napisanego w innych językach programowania niż JavaScript, w aplikacjach front-endowych, co daje olbrzymie możliwości
	\end{adjustwidth}
	\subsection*{profile:}
\begin{adjustwidth}{2cm}{}
	Szkolenie adresowane jest dla osób, które znają podstway języka JavaScript oraz dowolngo innego języka programowania
\end{adjustwidth}
	\subsection*{duration:}
\begin{adjustwidth}{2cm}{}
	2
\end{adjustwidth}

	\subsection*{list:}
\begin{adjustwidth}{1.5cm}{}
	\begin{itemize}










	\end{itemize}
\end{adjustwidth}

	\subsection*{skills:}
\begin{adjustwidth}{2cm}{}
	undefined
\end{adjustwidth}

\newpage


    
	\section{JavaScript dla programistów Java}

	\subsection*{description:}
	\begin{adjustwidth}{2cm}{}
		Cytująć klasyka - 'Java is to JavaScript as Car is to Carpet'. 
    Jest to szkolenie profilowane, dla osób ze świata Javy, które chciałby się dowiedzieć w jaki sposób działa JavaScript i jak w nim wygodnie pisać.
    W czasie tego szkolenia uczestnicy dostaną solidną dawkę wiedzy na temat mechaniki JavaScriptu, dzięki czemu język ten przestanie być (aż tak) upierdliwy.
	\end{adjustwidth}
	\subsection*{profile:}
\begin{adjustwidth}{2cm}{}
	
Szkolenie adresowane jest do programistów znających język Java.
\end{adjustwidth}
	\subsection*{duration:}
\begin{adjustwidth}{2cm}{}
	3
\end{adjustwidth}

	\subsection*{list:}
\begin{adjustwidth}{1.5cm}{}
	\begin{itemize}








	\end{itemize}
\end{adjustwidth}

	\subsection*{skills:}
\begin{adjustwidth}{2cm}{}
	
Po ukończonym szkoleniu uczestnicy będą się czuć znacznie pewniej w kodzie JavaScript, będą znali jego mechanike i będą wiedzieć w jaki sposób unikać typowych błędów

\end{adjustwidth}

\newpage


    
	\section{Progresywne przepisywanie aplikacji, zmniejszanie długu technologicznego i zasady clean code}

	\subsection*{description:}
	\begin{adjustwidth}{2cm}{}
		Celem szkolenia jest nauczenie uczestników w jaki sposób zabrać się do migracji legacy aplikacji.
     Szkolenie dotyczyć będzie również zasady czystego kodu, które pozwolą tworzyć aplikacje o znacznie wyższej jakości. 
     W czasie zajęć poruszane będą również tematy dotyczące długu technologicznego oraz techniki na jego minimalizację.

	\end{adjustwidth}
	\subsection*{profile:}
\begin{adjustwidth}{2cm}{}
	
Od uczestników szkolenia wymagana jest umiejętność programowania w dowolnym języku.

\end{adjustwidth}
	\subsection*{duration:}
\begin{adjustwidth}{2cm}{}
	2
\end{adjustwidth}

	\subsection*{list:}
\begin{adjustwidth}{1.5cm}{}
	\begin{itemize}










	\end{itemize}
\end{adjustwidth}

	\subsection*{skills:}
\begin{adjustwidth}{2cm}{}
	
Uczestnicy po ukończonym szkoleniu będą znać zasady pomagające w łatwiejszym migrowaniu aplikacji.

\end{adjustwidth}

\newpage


    
	\section{GraphQL}

	\subsection*{description:}
	\begin{adjustwidth}{2cm}{}
		
Celem szkolenia jest przekazanie wiedzy dotyczącej języka zapytań GraphQL będącego alternatywą dla protokołu REST. Uczestnicy po szkoleniu będą w stanie użyć tej technologii w swoich projektach. Szkolenie będzie dotyczyło implementacji GraphQL w języku JavaScript, niemniej jednak sama technologia może być również użyta z innymi językami takimi jak Java, C#, Go i wieloma innymi.
	\end{adjustwidth}
	\subsection*{profile:}
\begin{adjustwidth}{2cm}{}
	
Szkolenie adresowane jest do programistów znających dowolny język programowania.
\end{adjustwidth}
	\subsection*{duration:}
\begin{adjustwidth}{2cm}{}
	8
\end{adjustwidth}

	\subsection*{list:}
\begin{adjustwidth}{1.5cm}{}
	\begin{itemize}










	\end{itemize}
\end{adjustwidth}

	\subsection*{skills:}
\begin{adjustwidth}{2cm}{}
	
Po ukończonym szkoleniu uczestnicy będą posiadać wiedzę na temat tego w jaki sposób wykorzystać język zapytań GraphQL w aplikacjach webowych. Jak tworzyć odpowiednie zapytania, w jaki sposób dbać o bezpieczeństwo i jakich narzędzi używać przy implementowaniu aplikacji opierających się o GraphQL’a.



\end{adjustwidth}

\newpage


    
	\section{REST - od nowicjusza do geniusza}

	\subsection*{description:}
	\begin{adjustwidth}{2cm}{}
		W czasie szkolenia uczestnicy dowiedzą się na jakich zasadach dziala architektura REST, jakie są jej założenia i w jaki sposób tworzyć poprawne API oparte o tę architekturę.
	\end{adjustwidth}
	\subsection*{profile:}
\begin{adjustwidth}{2cm}{}
	
Od uczestników szkolenia wymagana jest umiejętność programowania w dowolnym języku.

\end{adjustwidth}
	\subsection*{duration:}
\begin{adjustwidth}{2cm}{}
	1
\end{adjustwidth}

	\subsection*{list:}
\begin{adjustwidth}{1.5cm}{}
	\begin{itemize}






	\end{itemize}
\end{adjustwidth}

	\subsection*{skills:}
\begin{adjustwidth}{2cm}{}
	
Uczestnicy po ukończonym szkoleniu będą znać zasady architektury REST.
\end{adjustwidth}

\newpage


    
	\section{regExp - wyrażenia regularne, zostań mistrzem!}

	\subsection*{description:}
	\begin{adjustwidth}{2cm}{}
		
Celem szkolenia jest nauczenie uczestników w jaki sposób osiągnąć mistrzostwo w tworzeniu wyrażeń regularnych. Szkolenie jest pełne ćwiczeń dzięki czemu każdy uczestnik będzie w stanie przećwiczyć nabytą wiedzę
	\end{adjustwidth}
	\subsection*{profile:}
\begin{adjustwidth}{2cm}{}
	
Szkolenie adresowane jest do programistów w dowolnych językach programownia.
\end{adjustwidth}
	\subsection*{duration:}
\begin{adjustwidth}{2cm}{}
	1
\end{adjustwidth}

	\subsection*{list:}
\begin{adjustwidth}{1.5cm}{}
	\begin{itemize}





	\end{itemize}
\end{adjustwidth}

	\subsection*{skills:}
\begin{adjustwidth}{2cm}{}
	
Po ukończonym szkoleniu uczestnicy będą potrafili tworzyć zaawansowane wyrażenia regularne
\end{adjustwidth}

\newpage


    
	\section{Business Intelligence, wizualizacja big data w aplikacjach internetowych}

	\subsection*{description:}
	\begin{adjustwidth}{2cm}{}
		Tworzenie wizualizacji danych jest skomplikowanym zadaniem, które wymaga nie tylko umiejętności przetworzenia początkowych zbiorów danych, do kształtu, który pragniemy zaprezentować, ale również wykorzystanie odpowiednich narzędzi do wizualizacji danych.
Strony WWW oraz aplikacje internetowe, są to doskonałe miejsca prezentacji danych a język do tworzenia aplikacji - JavaScript jest bogaty w narzędzia nie tylko do Business Intelligence ale i do Machine Learning i innych dziedzin data science.
Celem szkolenia jest przekazanie wiedzy na temat wizualizacji danych za pomocą takich bibliotek jak d3.js, która daje największe możliwości jeśli chodzi o prezentowanie zależności między danymi. W czasie szkolenia poruszone zostaną również tematy związane z Machine Learning w ciągle niedocenianym języku jakim jest JavaScript.


	\end{adjustwidth}
	\subsection*{profile:}
\begin{adjustwidth}{2cm}{}
	
Od uczestników szkolenia wymagana jest umiejętność programowania w dowolnym języku.

\end{adjustwidth}
	\subsection*{duration:}
\begin{adjustwidth}{2cm}{}
	2
\end{adjustwidth}

	\subsection*{list:}
\begin{adjustwidth}{1.5cm}{}
	\begin{itemize}







	\end{itemize}
\end{adjustwidth}

	\subsection*{skills:}
\begin{adjustwidth}{2cm}{}
	
Uczestnicy po ukończonym szkoleniu będą znać techniki wizualizacji danych w przeglądarce i będą wiedzieć jak się dalej rozwijać w tym kierunku.


\end{adjustwidth}

\newpage


    
	\section{Sieci neuronowe i uczenie maszynowe z tensorflow.js i brain.js}

	\subsection*{description:}
	\begin{adjustwidth}{2cm}{}
		Sieci neuronowe i uczenie maszynowe są to tematy, które w ostatnich latach zdobyły olbrzymią popularność. W czasie szkolenia uczestnicy dowiedzą się w jaki sposób używać dostępnych bibliotek napisanych w JavaScript, aby w prosty sposób przeprowadzać skomplikowane operacje.


	\end{adjustwidth}
	\subsection*{profile:}
\begin{adjustwidth}{2cm}{}
	
Od uczestników szkolenia wymagana jest umiejętność programowania w dowolnym języku.

\end{adjustwidth}
	\subsection*{duration:}
\begin{adjustwidth}{2cm}{}
	2
\end{adjustwidth}

	\subsection*{list:}
\begin{adjustwidth}{1.5cm}{}
	\begin{itemize}









	\end{itemize}
\end{adjustwidth}

	\subsection*{skills:}
\begin{adjustwidth}{2cm}{}
	
Uczestnicy po ukończonym szkoleniu będą znać techniki wizualizacji danych w przeglądarce i będą wiedzieć jak się dalej rozwijać w tym kierunku.


\end{adjustwidth}

\newpage


    
	\section{Komunikacja z developmentem (dla biznesu)}

	\subsection*{description:}
	\begin{adjustwidth}{2cm}{}
		
Największym problemem w czasie wytwarzania oprogramowania jest komunikacja. Przez nieodpowiedni sposób porozumiewania się, proces twórczy wydłuża się w nieskończoność. I zarówno część biznesowa jak i deweloperska projektu jest bardzo niezadowolona z warunków pracy.
W czasie szkolenia uczestnicy dowiedzą się w jaki sposób komunikować się z deweloperami, jak uzyskać od nich informacje, które są istotne w kontekście biznesu. 
Uczestnicy będą mieli szansę zrozumieć czemu tak często potrzeby programistów są zupełnie inne niż założone przez management, co wpływa na to, że zaczynają mniej wydajnie pracować czy nawet odchodzą z firmy. W czasie szkolenia przekazana zostanie również wiedza na temat programowania jako konceptu dzięki czemu komunikacja z programistami będzie dużo łatwiejsza.
	\end{adjustwidth}
	\subsection*{profile:}
\begin{adjustwidth}{2cm}{}
	Szkolenie będzie prowadzone od absolutnych podstaw. Języki programowania będą tłumaczone w analogii do języków naturalnych. Umiejętność programowania nie jest wymagana.
\end{adjustwidth}
	\subsection*{duration:}
\begin{adjustwidth}{2cm}{}
	2
\end{adjustwidth}

	\subsection*{list:}
\begin{adjustwidth}{1.5cm}{}
	\begin{itemize}













	\end{itemize}
\end{adjustwidth}

	\subsection*{skills:}
\begin{adjustwidth}{2cm}{}
	
Uczestnicy po ukończonym szkoleniu będą lepiej rozumieć potrzeby programistów, będą wiedzieli w jaki sposób lepiej się z nimi komunikować a także poznają slang, którym się posługują. 
\end{adjustwidth}

\newpage

